\chapter{Úvod}
%Průmysl 4.0
Žijeme v~době čtvrté průmyslové revoluce. Tato revoluce se projevuje rozšířenou automatizací a digitalizací továren. Továrny potenciálně generují obrovské množství různých dat, které stačí začít sbírat a zpracovávat. Tato data lze využít k~zefektivnění procesů výroby, a tím zvýšení zisků, což je pro podnikatelský sektor klíčové. Jedná se tedy o~jeden z~důvodů vzestupu Průmyslu 4.0 v~posledních letech. Jednou z~rozvíjejících se oblastí tohoto odvětví je také sledování strojů a jejich prediktivní údržba. Poruchy na klíčových zařízeních mohou způsobit několikadenní odstávky celé výrobní linky, a tím pádem znatelné finanční škody. Prediktivní údržba a monitorování strojů v~reálném čase může těmto situacím zabránit. Technologie umožňující prediktivní údržbu je založena na datech získaných ze senzorů umístěných na průmyslových strojích. Tyto senzory měří různé fyzikální veličiny, nejčastěji vibrace či teplotu. Pomocí matematických analýz lze následně vysledovat stav stroje, odhalit jeho nestandardní chování a faktory vedoucí k~poruše, a tím poruchu odhalit či jí předcházet. Tato metoda vyžaduje zpracování obrovského množství dat.

Systém navržený a implementovaný v~této bakalářské práci se opírá o~existující hardwarové řešení partnerské firmy zabývající se monitorováním, prediktivní analýzou a údržbou průmyslových strojů. Technologie firmy, včetně prototypu zařízení poskytnutého pro účely vypracování této bakalářské práce, jsou popsány dále v~textu. Na vývoji prototypu zařízení jsem se aktivně podílel. Hardware zařízení je navržen tak, aby cenově zpřístupnil možnost monitorování strojů a predikci poruch nejen pro velké, ale i pro menší firmy. Cílem této práce je navrhnout a implementovat software pro tuto jednotku a serverový software schopný s ní spolupracovat. Celý sytém umožňuje rozsáhlé nasazení monitorovacích zařízení v~krátkém časovém úseku. Klíčovým aspektem je proto snadná softwarová příprava měřících jednotek, jejich rychlé zavedení do systému a spuštění. Další důležitou součástí je možnost škálovatelnosti serverové části, zpracovávání velkého množství naměřených hodnot a integrace celého systému u~firem. 

V~kapitole \ref{kap:industry4.0} je diskutována nová generace průmyslu, důležitost využití velkých dat v~této oblasti a existující řešení pro monitorování průmyslových strojů. Informace týkající se sběru, zpracování a zobrazení velkých dat, které byla využita při návrhu systému, jsou shrnuty v~kapitole \ref{kap:bigData}. Možnost zápisu, uchování a přístupu k~obrovskému množství dat je jednou z~nejdůležitějších a nejkritičtějších částí práce s~velkými daty. Této tematice je proto věnována samostatná kapitola \ref{kap:storage}, popisující specifické požadavky na úložiště, základní principy často využívaných technologií a také příklady některých úložišť velkých dat. Základní požadavky a návrh celého systému obsahuje kapitola \ref{kap:design}. Implementace systému a zajímavé úseky softwarové části práce popisuje kapitola \ref{kap:implementation}. Testování jednotlivých částí systému je obsaženo v~kapitole \ref{kap:test}.
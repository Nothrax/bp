\chapter{Závěr}
Cílem této práce bylo navrhnout a implementovat systém pro sběr a zpracování dat ze senzorů dostupných v~průmyslových provozech. Výsledný systém obsahuje serverovou část, jednotku a komunikační protokoly pro zasílání dat. Serverový systém je navržen a implementován tak, aby mohl operovat pouze na jednom zařízení a zvládal přitom zpracovávat data nejméně padesáti jednotek. Zároveň, pokud je potřeba počet jednotek navýšit, je možné systém rozložit na více zařízení a jednotlivé komponenty škálovat. Serverové aplikace byly vytvořeny v~jazyce C++ a jsou určeny pro operační systém Ubuntu server. Pro ukládání informací o~systému je využita relační databáze a jako úložiště měřených dat je použita časová databáze. Jednotka byla poskytnuta partnerskou firmou a skládá se z~počítače Raspberry Pi a rozšiřující desky s~konektory pro čtyři senzory vibrací. Rozšiřující deska vzorkuje data ze senzorů o~frekvenci 128000 Hz. Software jednotky, implementovaný v~této práci, je tedy schopný zpracovávat až půl milionu hodnot za vteřinu. Data jsou odesílána na serverovou část dvěma způsoby. První způsob využívá navržený komunikační protokol \ref{pic:udp_protocol} postavený na UDP. Zprávy obsahují filtrované, agregované hodnoty s~frekvencí až sto hodnot za vteřinu, sloužící pro odhalení poruchy. Druhý způsob využívá protokol \ref{pic:tcp_protocol} postavený na TCP. Zprávy potom obsahují bloky nezpracovaných dat a umožňují provádění pokročilých analýz na serveru. V obou případech mohou být zprávy podle nastavení jednotky zašifrované. K~zabezpečení UDP zpráv je využito metody XOR šifrování, k~zabezpečení TCP spojení je využito SSL. Software jednotky je implementován v~jazyce C++ a obsahuje i instalační skripty pro rychlou a bezzásahovou instalaci.


Základními požadavky na systém je jeho škálovatelnost, rychlost a modularita. Škálovatelnost serverové části je dosažena několika metodami popsanými v~\ref{sec:load_balancing}. Systém je implementován tak, aby veškeré komponenty bylo možné samostatně škálovat. Serverová část také není závislá na implementaci jednotky, data mohou přicházet i od jiného typu jednotek než té implementované v~této práci. Musí ovšem využívat navržené komunikační protokoly. Systém je také navržen na více typů senzorů. Na historických datech partnerské firmy byla ověřena schopnost systému upozornit uživatele při poruše stroje detekované překročením nastaveného limitu u~příchozích dat.

Přínos této práce je ve využití volně dostupných systémů, jako je ekosystém TICK, pro zpracování velkých dat s~levnými jednodeskovými zařízeními pro sběr dat, jako je například Raspberry Pi. Propojením volně dostupných knihoven a programů je možné vytvořit škálovatelný systém pro monitorování průmyslových strojů, na kterém je možné stavět další pokročilejší funkcionalitu. Takové řešení je v porovnání s dostupnými komerčními produkty, které jsou většinou šity na míru, řádově levnější a univerzálnější. To potvrzuje i snaha partnerské firmy o pokračování vývoje systému navrženého a implementovaného v této práci a zájmu zákazníků o aplikaci podobných systémů v praxi (ZKL, Škoda Auto a podobně).
